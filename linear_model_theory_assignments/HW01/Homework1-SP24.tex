\documentclass[11pt]{article}
\usepackage{amssymb}
\usepackage{amsthm,amssymb,amsmath,amsbsy}
\usepackage{amsfonts}
\usepackage{graphicx}
\usepackage{color}
\usepackage{bm}
\usepackage{tcolorbox}
\usepackage{hyperref}
\usepackage{enumitem,kantlipsum}
\pagestyle{plain}
\newcommand{\V}{\vspace{0.3in}}
\newcommand{\VV}{\vspace{0.1in}}

\newcommand{\Real}{\mbox{\Bbb R}}
\newcommand{\Natural}{\mbox{\Bbb N}}
\newcommand{\Integer}{\mbox{\Bbb Z}}

\newcommand{\ra}{\rightarrow}

%%% Soutir Bandyopadhyay's spacing %%%
\oddsidemargin -0.25in 
\textwidth 7in 
\headheight 0in 
\textheight 9in 
\topmargin -0.5in 

%%%%%%%%%%%%%%%%%%%%%%%%%%%%%
\usepackage[framemethod=TikZ]{mdframed}

% here are a whole bunch of useful macros to make the latexing easier. 
% also note the handy macros \bex and \eex if you want to get fancy ...

\input{ourDefinitions.tex}

\begin{document}
%
%

 { \large \textbf{MATH 531}  Homework 1  Spring 2025 }\\
%
%
\begin{enumerate}
%
%

%
\item 
Let  $\{\bx_1, \bx_2,\ldots , \bx_k \}$ be a set of $k$ orthogonal vectors in $\R^n$. That is $\bx_i^T \bx_j = 0$ for all $i\ne j$. Show that this set is linearly independent when  $k\le n$. Also explain why this problem does not make sense when $k >  n$ !

%
%
\item Suppose ${\cal V}_{n} \subset \R^n$ is a vector space. Prove the following results.
%
%
\begin{enumerate}
\item If  $\bm{x}\in {\cal V}_{n}$ and $\bm{x}\perp {\cal V}_{n}$ then $\bm{x}=\bm{0}$.
\item ${\cal V}_{n}^{\perp}=\{\bm{x}:\bm{x}\perp {\cal V}_{n}\}$ is a vector space.
\end{enumerate}

\item Let $\{\bm{x}_{1},\ldots,\bm{x}_{k}\}$ be a basis of a vector space $\mathcal{W}$. Then show that $\bm{y} \in \mathcal{W}^\perp$ if and only if  (aka iff) $\bm{y}\perp \bm{x}_{i},\ i=1,2,\ldots,k$.
%
%
\item Recall the Cauchy-Schwartz inequality  for two vectors , $\bx$ and $\by$  in $\R^n$
\[  | \bx^T \by |  \le \| \bx \| \| \by \| \]
use this inequality to show that the sample correlation coefficient between two data vectors is always in the range  $[-1, 1]$

\item Recall the projection of a vector $\by$ onto $\bx$ is given by  $\hat{\by}= \beta \bx$ with $\beta = (\bx^T \by)/ (\bx^T \b)$
Using the fact that $\| \by - \hat{\by} \| \ge 0 $ for all $\bx$ and $\by$ prove the Cauchy-Schwartz inequality 

\item Extra Credit  Who were Cauchy and Schwarz in the equality given in the previous problem?  If you could go to dinner with either one who would you choose?


\end{enumerate}
%
%
%
%
%
%
%
%
\end{document}