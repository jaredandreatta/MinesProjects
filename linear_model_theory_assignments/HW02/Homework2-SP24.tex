\documentclass[11pt]{article}
\usepackage{amssymb}
\usepackage{amsthm,amssymb,amsmath,amsbsy}
\usepackage{amsfonts}
\usepackage{graphicx}
\usepackage{color}
\usepackage{bm}
\usepackage{tcolorbox}
\usepackage{hyperref}
\usepackage{enumitem,kantlipsum}
\pagestyle{plain}
\newcommand{\V}{\vspace{0.3in}}
\newcommand{\VV}{\vspace{0.1in}}

\newcommand{\Real}{\mbox{\Bbb R}}
\newcommand{\Natural}{\mbox{\Bbb N}}
\newcommand{\Integer}{\mbox{\Bbb Z}}

\newcommand{\ra}{\rightarrow}

%%% Soutir Bandyopadhyay's spacing %%%
\oddsidemargin -0.25in 
\textwidth 7in 
\headheight 0in 
\textheight 9in 
\topmargin -0.5in 

%%%%%%%%%%%%%%%%%%%%%%%%%%%%%
\usepackage[framemethod=TikZ]{mdframed}

% here are a whole bunch of useful macros to make the latexing easier. 
% also note the handy macros \bex and \eex if you want to get fancy ...

%\newtheorem{theorem}{Theorem}
%\newtheorem{proposition}[theorem]{Proposition}
%\newtheorem{lemma}[theorem]{Lemma}
%\newtheorem{corollary}[theorem]{Corollary}
%\newtheorem{definition}[theorem]{Definition}
%\newtheorem{example}[theorem]{Example}
%\newtheorem{assumption}[theorem]{Assumption}
%\newtheorem{remark}[theorem]{Remark}
%\DeclareMathOperator*{\argmin}{\arg\!\min}

% defines spiffy box around examples


% a handy macro to make a boldfaced 1  for vectors 
\newcommand{\ones}{\mbox{\boldmath $1$}}
\newcommand{\bdot}{ {\Large {\color{gray40} $\bullet$} }  }

\DeclareMathOperator*{\argmin}{\arg\!\min}

\newcommand{\il}[1]{\begin{itemize}  #1 \end{itemize}}

\newcommand{\T}{^{\rm T}}
\newcommand{\Cov}{{\rm Cov}}
\newcommand{\Cor}{{\rm Cor}}
\newcommand{\Var}{{\rm Var}}
\newcommand{\complex}{\mathbb{C}}
\newcommand{\real}{\mathbb{R}}
\newcommand{\integer}{\mathbb{Z}}
\newcommand{\indicator}{\mathbbm{1}}
\newcommand{\E}{\mathbb{E}}
\def\T{^{\rm T}}
\newcommand{\var}{var}
\newcommand{\cov}{cov}
\newcommand{\diag}{diag}
\newcommand{\tr}{tr}
\newcommand{\GP}{GP}
\newcommand{\avg}{avg}
\newcommand{\trace}{trace}
\newcommand{\blockdiag}{blockdiag}
\newcommand{\sign}{sign}
\newcommand{\knots}{\mathcal{Q}}
\newcommand{\grid}{\mathcal{G}}
\newcommand{\knot}{\mathbf{q}}
\newcommand{\likelihood}{\mathcal{L}}
\newcommand{\matern}{\mathcal{M}}
\newcommand{\node}{\mathcal{N}}
\newcommand{\normal}{\mathcal{N}}
\newcommand{\order}{\mathcal{O}}
\newcommand{\modu}{\mathcal{T}}
\renewcommand{\prec}{\boldsymbol{\Lambda}}
\newcommand{\pprec}{\widetilde{\boldsymbol{\Lambda}}}
\newcommand{\domain}{\mathcal{D}}
\newcommand{\pp}{\omega}
\newcommand{\locs}{\mathcal{S}}
\newcommand{\sphere}{S}

\newcommand{\ba}{\mathbf{a}}
\newcommand{\bb}{\mathbf{b}}
\newcommand{\bc}{\mathbf{c}}
\newcommand{\bd}{\mathbf{d}}
\newcommand{\be}{\mathbf{e}}
\newcommand{\beff}{\mathbf{f}}
\newcommand{\bbf}{\mathbf{f}} % because \bf means boldface!
\newcommand{\bg}{\mathbf{g}}
\newcommand{\bh}{\mathbf{h}}
\newcommand{\bi}{\mathbf{i}}
\newcommand{\bj}{\mathbf{j}}
\newcommand{\bk}{\mathbf{k}}
\newcommand{\bl}{\mathbf{l}}
\newcommand{\bell}{\boldsymbol{\ell}}
\newcommand{\bbm}{\mathbf{m}}
\newcommand{\bn}{\mathbf{n}}
\newcommand{\bo}{\mathbf{o}}
\newcommand{\bp}{\mathbf{p}}
\newcommand{\bq}{\mathbf{q}}
\newcommand{\br}{\mathbf{r}}
\newcommand{\bs}{\mathbf{s}}
\newcommand{\bt}{\mathbf{t}}
\newcommand{\bu}{\mathbf{u}}
\newcommand{\bv}{\mathbf{v}}
\newcommand{\bw}{\mathbf{w}}
\newcommand{\bx}{\mathbf{x}}
\newcommand{\by}{\mathbf{y}}
\newcommand{\bz}{\mathbf{z}}

\newcommand{\vep}{\varepsilon}
\newcommand{\vphi}{\varphi}

\newcommand{\sI}{\mathscr{I}}
\newcommand{\sR}{\mathscr{R}}
\newcommand{\sP}{\mathscr{P}}

\newcommand{\cD}{{\cal D}}
\newcommand{\cH}{{\cal H}}
\newcommand{\bA}{\mathbf{A}}
\newcommand{\bB}{\mathbf{B}}
\newcommand{\bC}{\mathbf{C}}
\newcommand{\bD}{\mathbf{D}}
\newcommand{\bG}{\mathbf{G}}
\newcommand{\bH}{\mathbf{H}}
\newcommand{\bI}{\mathbf{I}}
\newcommand{\bK}{\mathbf{K}}
\newcommand{\bL}{\mathbf{L}}
\newcommand{\bM}{\mathbf{M}}
\newcommand{\bP}{\mathbf{P}}
\newcommand{\bQ}{\mathbf{Q}}
\newcommand{\bR}{\mathbf{R}}
\newcommand{\bS}{\mathbf{S}}
\newcommand{\bT}{\mathbf{T}}
\newcommand{\bU}{\mathbf{U}}
\newcommand{\bV}{\mathbf{V}}
\newcommand{\bW}{\mathbf{W}}
\newcommand{\bX}{\mathbf{X}}
\newcommand{\bY}{\mathbf{Y}}
\newcommand{\bZ}{\mathbf{Z}}
\newcommand{\bomega}{\boldsymbol{\omega}}
\newcommand{\bbeta}{\boldsymbol{\beta}}
\newcommand{\bepsilon}{\boldsymbol{\epsilon}}
\newcommand{\bphi}{\boldsymbol{\phi}}
\newcommand{\bPhi}{\boldsymbol{\Phi}}
\newcommand{\blambda}{\boldsymbol{\lambda}}
\newcommand{\btheta}{\boldsymbol{\theta}}
\newcommand{\bvep}{\boldsymbol{\varepsilon}}
\newcommand{\bmu}{\boldsymbol{\mu}}
\newcommand{\bnu}{\boldsymbol{\nu}}
\newcommand{\bpsi}{\boldsymbol{\psi}}


\newcommand{\bfzero}{\mathbf{0}}
\newcommand{\bfalpha}{\boldsymbol{\alpha}}
\newcommand{\bfgamma}{\boldsymbol{\gamma}}
\newcommand{\bfmu}{\boldsymbol{\mu}}
\newcommand{\bfxi}{\boldsymbol{\xi}}
\newcommand{\bftheta}{\boldsymbol{\theta}}
\newcommand{\bfeta}{\boldsymbol{\eta}}
\newcommand{\bfnu}{\boldsymbol{\nu}}
\newcommand{\bfrho}{\boldsymbol{\rho}}
\newcommand{\bfdelta}{\boldsymbol{\delta}}
\newcommand{\bfkappa}{\boldsymbol{\kappa}}
\newcommand{\bfbeta}{\boldsymbol{\beta}}
\newcommand{\bfepsilon}{\boldsymbol{\epsilon}}
\newcommand{\bftau}{\boldsymbol{\tau}}
\newcommand{\bfomega}{\boldsymbol{\omega}}
\newcommand{\bfpi}{\boldsymbol{\pi}}
\newcommand{\bfpsi}{\boldsymbol{\psi}}
\newcommand{\bfSigma}{\boldsymbol{\Sigma}}
\newcommand{\bfGamma}{\boldsymbol{\Gamma}}
\newcommand{\bfLambda}{\boldsymbol{\Lambda}}
\newcommand{\bfPsi}{\boldsymbol{\Psi}}
\newcommand{\bfOmega}{\boldsymbol{\Omega}}

\newcommand{\im}{{i_1,\ldots,i_m}}
\newcommand{\jm}{{j_1,\ldots,j_m}}
\newcommand{\jmp}{{j_1,\ldots,j_{m+1}}}
\newcommand{\jmm}{{j_1,\ldots,j_{m-1}}}
\newcommand{\jk}{{j_1,\ldots,j_k}}
\newcommand{\jl}{{j_1,\ldots,j_l}}
\newcommand{\jlp}{{j_1,\ldots,j_{l+1}}}
\newcommand{\jM}{{j_1,\ldots,j_M}}

\newcommand{\evol}{\mathcal{E}}
\newcommand{\levol}{\mathbf{E}}

\newcommand{\xp}{\mathbf{\widetilde{x}}} 
\newcommand{\yp}{\mathbf{\widetilde{y}}}
\newcommand{\xf}{\mathbf{\widehat{x}}}
\newcommand{\yf}{\mathbf{\widehat{y}}}

\newcommand{\Lp}{\mathbf{\widetilde{L}}}  
\newcommand{\Lf}{\mathbf{\widehat{L}}}  
\newcommand{\Sp}{\mathbf{\widetilde{S}}}  
\newcommand{\Sf}{\mathbf{\widehat{S}}}  

\newcommand{\ap}{\boldsymbol{\widetilde{\mu}}}     
\newcommand{\Pp}{\boldsymbol{\widetilde{\Sigma}}}  
\newcommand{\af}{\boldsymbol{\widehat{\mu}}}     
\newcommand{\Pf}{\boldsymbol{\widehat{\Sigma}}}  

\newcommand{\kronecker}{\raisebox{1pt}{\ensuremath{\:\otimes\:}}}

\newcommand{\bzero}{\boldsymbol{0}}
\newcommand{\bone}{\boldsymbol{1}}


\begin{document}
%
%
\begin{center}
\begin{tabular}{lcr}
\textbf{MATH 531} & \hskip0.6in\textbf{Homework 2} & \hskip0.6in\textbf{Spring 2024}\\
& \hskip0.6in\textbf{(Due: 29th January, Monday)} &
\end{tabular}
\end{center}
%
%
\begin{enumerate}
%
%

%
\item 
Let  $\{\bz_1, \bz_2,\ldots , \bz_k \}$ be a set of $k$ orthonormal vectors in ${\Bbb R}^n$  that span a subspace $\cal W$ and let $Z$ be a matrix formed by taking these as column vectors. 
\[ Z=  [\bz_1, \bz_2,\ldots , \bz_k  ] \]
( Note in R this would be the {\tt cbind} function to form the vectors into a matrix. )
Show that  $Z^T Z = I $ where $I$ is the identity matrix. What is the dimension  of this identity?  (Subsequently  when it is important we will notate the identity matrix as $I_m$ where $m$ refers to the dimension. )

\item Show that   the  trace of the matrix $ZZ^T$ is $k$.  Under what circumstances will $ZZ^T = I$?

\item  
Define: The {\it projection} of $\by \in {\Bbb R}^n$ onto a subspace $\cal W$ as the vector $\bu$ such that  $\by = \bu + \bv$, $\bu \in \cal W$ and $\bv  \in {\cal W}^\perp $.  

From lecture we know  that if a projection exists it is unique -- so we can talk about  {\it the} projection. 

(This is not exactly how projection is defined in the course reference notes but is how it was presented in class lecture, followed by a proof of existence and uniqueness. )

From Problem (1) define the $n \times n$ matrix  $P=  ZZ^T$. Show that for any $\by$, $\bu = P \by$ is the projection. 

Show that $(I-P) \by$ is the projection onto ${\cal W}^\perp $.

\item Let $X$ be an  $n \times k$ matrix   where the columns are linearly independent. Show that $X^T X$ has full rank ($k$) and show that $P= X (X^T X)^{-1} X^T $ is a projection onto the subspace spanned by the columns of $X$. 

\item Classical treatments of linear algebra tend to present the singular value decomposition (svd) at the end of a course -- if at all. However, it is an extremely useful way to categorize all matrices and hence all linear maps between 
${\Bbb R}^k $ and ${\Bbb R}^n$.  Let $X$ be any $n \times k $ matrix, then 
\[ X = U D  V^T \]  with 
\begin{itemize}
\item  $U$ an $n \times k $ matrix where the columns form an orthonormal basis ( $U^TU = I_k$) 
\item  $D$ is a  $k \times k $ diagonal matrix with nonnegative elements (some of the $D$ can be zero and the convention is to sort these in descending order.) 
\item  $ V$ is an $k \times k $ orthonormal matrix  ($ V^T  V = I _k$) 
\end{itemize}
(The proof of this is long and involved and we will just take this result on trust!)

If  $A$ is a square matrix and invertible show  that  $A^{-1} = V D^{-1} U^T $.

Show that  $X^T X$ = $  V D^2 V^T$. 

Based on this previous result explain how to construct a G-inverse for $X^T X$, that is identify a matrix, A, such that
$X^T X( A )X^T X = X^T X$.


\item When was the singular value decomposition invented? If possible identify its inventor(s)? 
Is it difficult to  compute the SVD of a matrix in R?

\end{enumerate}
%
%
%
%
%
%
%
%
\end{document}