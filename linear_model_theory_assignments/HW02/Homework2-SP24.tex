\documentclass[11pt]{article}
\usepackage{amssymb}
\usepackage{amsthm,amssymb,amsmath,amsbsy}
\usepackage{amsfonts}
\usepackage{graphicx}
\usepackage{color}
\usepackage{bm}
\usepackage{tcolorbox}
\usepackage{hyperref}
\usepackage{enumitem,kantlipsum}
\pagestyle{plain}
\newcommand{\V}{\vspace{0.3in}}
\newcommand{\VV}{\vspace{0.1in}}

\newcommand{\Real}{\mbox{\Bbb R}}
\newcommand{\Natural}{\mbox{\Bbb N}}
\newcommand{\Integer}{\mbox{\Bbb Z}}

\newcommand{\ra}{\rightarrow}

%%% Soutir Bandyopadhyay's spacing %%%
\oddsidemargin -0.25in 
\textwidth 7in 
\headheight 0in 
\textheight 9in 
\topmargin -0.5in 

%%%%%%%%%%%%%%%%%%%%%%%%%%%%%
\usepackage[framemethod=TikZ]{mdframed}

% here are a whole bunch of useful macros to make the latexing easier. 
% also note the handy macros \bex and \eex if you want to get fancy ...

\input{ourDefinitions.tex}

\begin{document}
%
%
\begin{center}
\begin{tabular}{lcr}
\textbf{MATH 531} & \hskip0.6in\textbf{Homework 2} & \hskip0.6in\textbf{Spring 2024}\\
& \hskip0.6in\textbf{(Due: 29th January, Monday)} &
\end{tabular}
\end{center}
%
%
\begin{enumerate}
%
%

%
\item 
Let  $\{\bz_1, \bz_2,\ldots , \bz_k \}$ be a set of $k$ orthonormal vectors in ${\Bbb R}^n$  that span a subspace $\cal W$ and let $Z$ be a matrix formed by taking these as column vectors. 
\[ Z=  [\bz_1, \bz_2,\ldots , \bz_k  ] \]
( Note in R this would be the {\tt cbind} function to form the vectors into a matrix. )
Show that  $Z^T Z = I $ where $I$ is the identity matrix. What is the dimension  of this identity?  (Subsequently  when it is important we will notate the identity matrix as $I_m$ where $m$ refers to the dimension. )

\item Show that   the  trace of the matrix $ZZ^T$ is $k$.  Under what circumstances will $ZZ^T = I$?

\item  
Define: The {\it projection} of $\by \in {\Bbb R}^n$ onto a subspace $\cal W$ as the vector $\bu$ such that  $\by = \bu + \bv$, $\bu \in \cal W$ and $\bv  \in {\cal W}^\perp $.  

From lecture we know  that if a projection exists it is unique -- so we can talk about  {\it the} projection. 

(This is not exactly how projection is defined in the course reference notes but is how it was presented in class lecture, followed by a proof of existence and uniqueness. )

From Problem (1) define the $n \times n$ matrix  $P=  ZZ^T$. Show that for any $\by$, $\bu = P \by$ is the projection. 

Show that $(I-P) \by$ is the projection onto ${\cal W}^\perp $.

\item Let $X$ be an  $n \times k$ matrix   where the columns are linearly independent. Show that $X^T X$ has full rank ($k$) and show that $P= X (X^T X)^{-1} X^T $ is a projection onto the subspace spanned by the columns of $X$. 

\item Classical treatments of linear algebra tend to present the singular value decomposition (svd) at the end of a course -- if at all. However, it is an extremely useful way to categorize all matrices and hence all linear maps between 
${\Bbb R}^k $ and ${\Bbb R}^n$.  Let $X$ be any $n \times k $ matrix, then 
\[ X = U D  V^T \]  with 
\begin{itemize}
\item  $U$ an $n \times k $ matrix where the columns form an orthonormal basis ( $U^TU = I_k$) 
\item  $D$ is a  $k \times k $ diagonal matrix with nonnegative elements (some of the $D$ can be zero and the convention is to sort these in descending order.) 
\item  $ V$ is an $k \times k $ orthonormal matrix  ($ V^T  V = I _k$) 
\end{itemize}
(The proof of this is long and involved and we will just take this result on trust!)

If  $A$ is a square matrix and invertible show  that  $A^{-1} = V D^{-1} U^T $.

Show that  $X^T X$ = $  V D^2 V^T$. 

Based on this previous result explain how to construct a G-inverse for $X^T X$, that is identify a matrix, A, such that
$X^T X( A )X^T X = X^T X$.


\item When was the singular value decomposition invented? If possible identify its inventor(s)? 
Is it difficult to  compute the SVD of a matrix in R?

\end{enumerate}
%
%
%
%
%
%
%
%
\end{document}