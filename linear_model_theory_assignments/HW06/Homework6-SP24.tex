\documentclass[10pt]{report}
\nofiles
\usepackage{amsmath}
\usepackage{graphicx,psfrag,epsf}
\usepackage{enumerate}
%\usepackage{natbib}
\usepackage{float}
\usepackage{color}
\usepackage{graphicx}
\usepackage{verbatim}
\usepackage{arydshln}

\usepackage{comment}
\usepackage[framemethod=TikZ]{mdframed}
\usepackage{lipsum}

\input{/Users/nychka/Home/Tex/Teaching/ColorsFromR.tex}
\input{/Users/nychka/Home/Tex/Teaching/ourDefinitions.tex}

\setlength{\textwidth}{5.5in}
\setlength{\evensidemargin}{0.50in}
\setlength{\oddsidemargin}{0.50in} 
\setlength{\textheight}{8.75in}
\setlength{\topmargin}{0.00in}
\setlength{\parskip}{.25in}
\setlength{\parindent}{.25in}

\renewcommand{\baselinestretch}{1.0}



\begin{document}
\vspace*{-1in}
\noindent
{\LARGE  \bf  \sc  MATH 531  Homework 6  \\
  }
%{\large \it Colorado School of Mines}
\noindent
{\Large \bf   Fisher information  \\ \today} 
\ \\
{\color{orange3} \hrule  }
\ \\
Each subsection counts for 10 points  -- 50 points total and the extra credit counts for 5.  
\begin{enumerate}
%%%%%%%%%%%%%%%%%%%%%%%%%%%%%%%%%%%%%%%%%%%%
%PROBLEM 1
%%%%%%%%%%%%%%%%%%%%%%%%%%%%%%%%%%%%%%%%%%%%
\item  
 Assume a linear model 
\[  \by = X \beta + \be \]
where $\by$ is an vector of length $n$ and $X$ is a full rank,
 $n\times p$ matrix. $\be_i$ are independent $N(0,\sigma^2\Omega)$. Also suppose that $\beta_T$ is ``true" value for $\beta$ and $\hat \beta$ the Generalized Least Squares (GLS) /MLE estimate and $\hat{\sigma}$ the MLE for $\sigma$. Here $\Omega$ is a known correlation matrix. 

\begin{enumerate}
 \item Identify the formulas for the MLEs $\hat \beta$ and  $\hat{\sigma}_{MLE}$.
 \item Derive the Fisher information for this model when evaluated at the true values of $\beta$ and $\sigma^2$.
 (You should reparametrize $\sigma^2$ as $\omega$ as in the lectures to make derivatives easier. )
 
 \item Do the GLS estimates for $\beta$ have a covariance matrix that achieves the Cramer-Rao lower bound (
  the inverse Fisher information matrix)?
 \item The  GLS estimate for $\sigma^2$  based on previous homework differs slightly from the MLE but it is unbiased. 
 \[ \hat{\sigma}^2_{GLS} =  \frac{1}{(n-p)} (\by - X\hat{\beta})^T \Omega^{-1} (\by - X\hat{\beta}) \] 
 and has variance 
$ \frac{2\sigma^4}{(n-p)}$
 
 Show that that the variance of this estimate achieves the Cramer-Rao lower bound in the limit as $n \rightarrow \infty$.
 \end{enumerate}
 
%%%%%%%%%%%%%%%%%%%%%%%%%%%%%%%%%%%%%%%%%%%%
%PROBLEM 2
%%%%%%%%%%%%%%%%%%%%%%%%%%%%%%%%%%%%%%%%%%%%


\item 
 Now supposed that $\Omega$ above depends on an additional statistical parameter, say $\Omega(\alpha)$. 
 (This is common in models for spatial and time series data.)
 Derive the Fisher Information for this extended model. Explain why finding the GLS estimate in this case is problematic.   

 \end{enumerate}
 
\end{document}




